
\documentclass[letterpaper, reqno,11pt]{article}
\usepackage[margin=1.0in]{geometry}
\usepackage{color,latexsym,amsmath,amssymb}
\usepackage{fancyhdr}
\usepackage{amsthm}
\usepackage{mathtools}
\usepackage{tikz}
\usepackage{float}
\usepackage{centernot}
\usepackage{subcaption}
\usepackage{extarrows}
\usetikzlibrary{hobby}
\usetikzlibrary{shapes.multipart}
\usepackage{pgfplots}
\pgfplotsset{compat=1.7}
\usetikzlibrary{arrows.meta}
\usepackage{cancel}

\newcommand{\RR}{\mathbb{R}}
\newcommand{\CC}{\mathbb{C}}
\newcommand{\ZZ}{\mathbb{Z}}
\newcommand{\QQ}{\mathbb{Q}}
\newcommand{\NN}{\mathbb{N}}
\def\upint{\mathchoice%
  {\mkern13mu\overline{\vphantom{\intop}\mkern7mu}\mkern-20mu}%
  {\mkern7mu\overline{\vphantom{\intop}\mkern7mu}\mkern-14mu}%
  {\mkern7mu\overline{\vphantom{\intop}\mkern7mu}\mkern-14mu}%
  {\mkern7mu\overline{\vphantom{\intop}\mkern7mu}\mkern-14mu}%
  \int}
\def\lowint{\mkern3mu\underline{\vphantom{\intop}\mkern7mu}\mkern-10mu\int}
\DeclareMathOperator{\card}{card}
\DeclareMathOperator{\Binomial}{Binomial}
\pagestyle{fancy}
\lhead{Math 321 Lecture 21}
\rhead{Yuchong Pan}
\begin{document}
\pagenumbering{arabic}
\title{Math 321 Lecture 21}
\author{Yuchong Pan}
\date{February 27, 2019}
\newtheorem{thm}{Theorem}
\newtheorem{defn}{Definition}
\newtheorem*{remark}{Remark}
\newtheorem{claim}{Claim}
\newtheorem{cor}{Corollary}
\newtheorem{lemma}{Lemma}
\newtheorem{prop}{Proposition}
\newtheorem{fact}{Fact}
\maketitle
%

\section{Fourier Series}

Consider the interval $[-\pi, \pi]$.

\begin{defn}
  \normalfont A function $T : [-\pi, \pi] \to \CC$ is called a {\bf trignometric polynomial} if $T$ is of the form
  \[ T = \alpha_0 + \sum_{k = 1}^n \left(\underbrace{\alpha_k}_\textbf{amplitude} \cos kx + \underbrace{\beta_k}_\textbf{amplitude} \sin kx\right), \]
  where $\underbrace{\alpha_k}_{0 \leq k \leq n}, \underbrace{\beta_k}_{1 \leq k \leq n} \in \CC$.

  Say $T$ has a {\bf frequency} $k \in \{ \pm 1, \ldots, \pm n \}$ if at least one of $\alpha_k$ or $\beta_k$ does not equal $0$.

  Alternatively, $T$ can also be expressed as
  \begin{equation} \label{eq:*} \tag{*}
    T(x) = \sum_{k = -n}^n a_k e^{ikx}, a_k \in \CC,
  \end{equation}
  because $e^{i\theta} = \cos \theta + i\sin \theta$ so $\cos\theta = \frac{e^{i\theta} + e^{-i\theta}}{2}, \sin\theta = \frac{e^{i\theta} - e^{-i\theta}}{2}$.
\end{defn}

Recall Weierstrass's second theorem: Any $f \in \mathcal C^{2\pi} =$ the space of complex-valued, $2\pi$-periodic, continuous functions on $[-\pi, \pi]$ can be uniformly approximated by a sequence of trignometric polynomials.

\medskip

\noindent {\bf Question:} Given a trignometric polynomial $T$ with unspecified coefficients $\alpha_k, \beta_k$ (or $a_k$), is it possible to obtain its frequencies and corresponding amplitudes?
\[ T(x) = \alpha_0 + \sum_{k = 1}^n (\alpha_k \cos kx + \beta_k \sin kx), \qquad \text{$n, \alpha_k, \beta_k$ unknown}. \]

\begin{fact} \label{fact:1}
  \normalfont Let $\{ 1, \cos mx, \sin mx : m = 1, 2, 3, \ldots \} = \mathcal F^*$.
  \begin{enumerate}
  \item For any two $f, g \in \mathcal F^*, f \neq g$,
    \[ \int_{-\pi}^\pi f(x) g(x) dx = 0. \]
    Note that
    \begin{align*}
      \cos mx \sin nx &= \frac{1}{2} (\sin (n + m)x + \sin (n - m)x), \\
      \cos mx \cos nx &= \frac{1}{2} (\cos (n + m)x + \cos (n - m)x).
    \end{align*}
  \item If $f = g$,
    \[ \int_{-\pi}^\pi 1^2 dx = 2\pi, \]
    \[ \int_{-\pi}^\pi \cos^2 mx dx = \int_{-\pi}^\pi (1 + \cos 2mx) dx = \pi = \int_{-\pi}^\pi \sin^2 mx dx, \qquad m = 1, 2, 3, \ldots. \]
  \end{enumerate}
\end{fact}

Integrate both sides on $[-\pi, \pi]$,
\[ \int_{-\pi}^\pi T(x) dx = \alpha_0 \int_{-\pi}^\pi d(x) + \sum_{k = 1}^n \left(\alpha_k \underbrace{\int_{-\pi}^\pi \cos kx dx}_{= 0} + \beta_k \underbrace{\int_{-\pi}^\pi \sin kx dx}_{= 0}\right). \]
Thus,
\[ \alpha_0 = \frac{1}{2\pi} \int_{-\pi}^\pi T(x) dx = \frac{1}{2\pi} \langle T, 1 \rangle. \]
To solve for $\alpha_m$, multiply both sides of \eqref{eq:*} by $\cos mx$:
\[ \int_{-\pi}^\pi T(x) \cos mx dx = \alpha_m \int_{-\pi}^\pi \cos^2 mx + \underbrace{\ldots}_\text{all remaining integrals equal to $0$ by Fact \ref{fact:1}}, \]
i.e.,
\begin{align*}
  \alpha_m &= \frac{1}{\pi} \int_{-\pi}^\pi T(x) \cos mx dx = \frac{1}{\pi} \langle T(\cdot), \cos(m \cdot) \rangle, \\
  \beta_m &= \frac{1}{\pi} \int_{-\pi}^\pi T(x) \sin mx dx = \frac{1}{\pi} \langle T(\cdot), \sin(m \cdot) \rangle.
\end{align*}
Here,
\[ \langle f, g \rangle = \int_{-\pi}^\pi f(x) g(x) dx. \]
We have just shown that for a trignometric polynomial $T$,
\[ T(x) = \underbrace{\frac{\langle T, 1 \rangle}{2\pi}}_{= \alpha_0} + \sum_{k = 1}^n \left(\underbrace{\frac{\langle T(\cdot), \cos (k \cdot) \rangle}{\pi}}_{= \alpha_k} \cos kx + \underbrace{\frac{\langle T(\cdot), \sin (k \cdot) \rangle}{\pi}}_{= \beta_k} \sin kx\right). \]

\begin{enumerate}
\item Is the representation unique?
\item Can we do this for larger classes of functions, say $\mathcal C^{2\pi}$?
\end{enumerate}

\begin{defn}
  \normalfont Let $f \in \mathcal C^{2\pi}$, or more generally $f \in \mathcal R[-\pi, \pi]$. Define
  \begin{equation} \label{eq:**} \tag{**}
    \left\{
    \begin{array}{lcl}
      \alpha_0 &=& \frac{\langle f, 1 \rangle}{2\pi} = \frac{1}{2\pi} \int_{-\pi}^\pi f(x) dx, \\
      \alpha_m &=& \frac{\langle f(\cdot), \cos(m \cdot) \rangle}{\pi} = \frac{1}{\pi} \int_{-\pi}^\pi f(x) \cos mx dx, \qquad m = 1, 2, \ldots, \\
      \beta_m &=& \frac{\langle f(\cdot), \sin(m \cdot) \rangle}{\pi} = \frac{1}{\pi} \int_{-\pi}^\pi f(x) \sin mx dx, \qquad m = 1, 2, \ldots.
    \end{array}
    \right.
  \end{equation}
  Note: $\alpha_m, \beta_m$ could be nonzero for infinitely many $m$ (unlike trignometric polynomials $T$).
\end{defn}

\noindent {\bf Question:} Is it possible that the {\bf Fourier series} of $f$, where $\alpha_k, \beta_k$ are defined in \eqref{eq:**}, defined by \fbox{$\alpha_0 + \sum_{k = 1}^\infty (\alpha_k \cos kx + \beta_k \sin kx)$}, converges for every $f \in \mathcal C^{2\pi}$ (or $\mathcal R[-\pi, \pi]$)?

If so, can it be true that
\begin{equation} \label{eq:1}
  f(x) = \alpha_0 + \sum_{k = 0}^\infty (\alpha_k \cos kx + \beta_k \sin kx), \qquad \forall x \in [-\pi, \pi].
\end{equation}

\begin{remark}
  \normalfont False, even for $f \in \mathcal C^{2\pi}$: there exists $f \in \mathcal C^{2\pi}$ such that the partial sums of the right-hand side of \eqref{eq:1} goes to $\infty$.

  Why does this not contradict WS $2^\text{nd}$ theorem?
\end{remark}

\end{document}
