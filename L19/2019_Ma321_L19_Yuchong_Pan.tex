
\documentclass[letterpaper, reqno,11pt]{article}
\usepackage[margin=1.0in]{geometry}
\usepackage{color,latexsym,amsmath,amssymb}
\usepackage{fancyhdr}
\usepackage{amsthm}
\usepackage{mathtools}
\usepackage{tikz}
\usepackage{float}
\usepackage{centernot}
\usepackage{subcaption}
\usepackage{extarrows}
\usetikzlibrary{hobby}
\usetikzlibrary{shapes.multipart}
\usepackage{pgfplots}
\pgfplotsset{compat=1.7}
\usetikzlibrary{arrows.meta}

\newcommand{\RR}{\mathbb{R}}
\newcommand{\CC}{\mathbb{C}}
\newcommand{\ZZ}{\mathbb{Z}}
\newcommand{\QQ}{\mathbb{Q}}
\newcommand{\NN}{\mathbb{N}}
\def\upint{\mathchoice%
  {\mkern13mu\overline{\vphantom{\intop}\mkern7mu}\mkern-20mu}%
  {\mkern7mu\overline{\vphantom{\intop}\mkern7mu}\mkern-14mu}%
  {\mkern7mu\overline{\vphantom{\intop}\mkern7mu}\mkern-14mu}%
  {\mkern7mu\overline{\vphantom{\intop}\mkern7mu}\mkern-14mu}%
  \int}
\def\lowint{\mkern3mu\underline{\vphantom{\intop}\mkern7mu}\mkern-10mu\int}
\DeclareMathOperator{\card}{card}
\DeclareMathOperator{\Binomial}{Binomial}
\pagestyle{fancy}
\lhead{Math 321 Lecture 19}
\rhead{Yuchong Pan}
\begin{document}
\pagenumbering{arabic}
\title{Math 321 Lecture 19}
\author{Yuchong Pan}
\date{February 15, 2019}
\newtheorem{thm}{Theorem}
\newtheorem{defn}{Definition}
\newtheorem*{remark}{Remark}
\newtheorem{claim}{Claim}
\newtheorem{cor}{Corollary}
\newtheorem{lemma}{Lemma}
\newtheorem{prop}{Proposition}
\maketitle
%

\section{Linear Functionals on Normed Vector Spaces}

\subsection{Definition and Examples}

Let $V$ be a vector space on $\RR$, equipped with a norm $\lVert \ \cdot \ \rVert_V$. $\lVert \ \cdot \ \rVert_V$ generates a metric on $V$ via $d(\mathbf v, \mathbf w) = \lVert \mathbf v - \mathbf w \rVert_V$. This allows to define \emph{continuous functions} on $X$.

\medskip

\noindent {\bf Recall:} A function $f : V \to \RR$ is continuous (by definition) if $f^{-1}(\text{open set in $\RR$})$ is open in $V$.

\begin{defn}
  \normalfont Say $T : V \to \RR$ is a {\bf continuous linear functional} on $V$ if
  \begin{enumerate}
  \item $T$ is linear: $T(\alpha \mathbf v + \beta \mathbf w) = \alpha T(\mathbf v) + \beta T(\mathbf w)$ for all $\mathbf v, \mathbf w \in V, \alpha, \beta \in \RR$;
  \item $T$ is continuous.
  \end{enumerate}
\end{defn}

\noindent {\bf Examples:}
\begin{enumerate}
\item $V = \RR^n$; what is a continuous linear functional on $V$?

  $T : \RR^n \xrightarrow{\text{linear}} \RR \Leftrightarrow \exists \mathbf a \in \RR^n \text{ s.t. } T(\mathbf x) = \sum_{j = 1}^n a_j x_j ~ \forall \mathbf x = (x_1, \ldots, x_n)$. These are continuous.

  \medskip

  \noindent {\bf Exercise:} Check that if $V$ is finite-dimensional, then any linear functional on $V$ is continuous.
\item Is this fact necessarily true if $V$ is infinite-dimensional?
  \begin{enumerate}
  \item
    \begin{align*}
      V &= \{ \text{all infinite sequences with all but finitely many entries $0$} \} \\
      &= \{ \mathbf x = (x_1, x_2, x_3, \ldots) : \exists N \geq 1 \text{ s.t. } x_j = 0 ~ \forall j \geq N \}.
    \end{align*}

    $V$ is infinite-dimensional because $\left\{ \mathbf e_j = (0, 0, \ldots, \underbrace{j}_\text{$j^\text{th}$ entry}, 0, \ldots) : j \geq 1 \right\}$ is an infinite linearly independent set in $V$.

    Define $T(\mathbf x) = \sum_{j = 1}^\infty x_j$. Recall that $\lVert \mathbf x \rVert_1 = \sum_{j = 1}^\infty |x_j|$. Note that $T$ is linear and $|T(\mathbf x)| \leq \sum_{j = 1}^\infty |x_j| = \lVert \mathbf x \rVert_1$.

    \begin{claim}
      \normalfont $T$ is continuous and linear with respect to $\lVert \ \cdot \ \rVert_1$.
    \end{claim}

    Define $\lVert \mathbf x \rVert^* = \sum_{j = 1}^\infty 2^{-j} |x_j|$. Note that $T(\mathbf e_j) = 1$ for every $j \geq 1$ and $\lVert \mathbf e_j \rVert^* = 2^{-j}$. This implies that
    $$ \frac{|T(\mathbf e_j)|}{\lVert \mathbf e_j \rVert^*} = \frac{1}{2^{-j}} = 2^j \xrightarrow{j \to \infty} \infty. $$
  \item $\mathcal P = \{ \text{all polynomials on $[0, 1]$} \}$, equipped with the sup norm.

    $\mathcal P$ is infinite-dimensional because $\left\{ \underbrace{x_n}_{= P_n} : n \geq 0 \right\}$ form a linearly independent set.

    Let $T : \mathcal P \to \RR$ be defined by $f \mapsto f'(1)$.

    \begin{claim}
      \normalfont $T$ is linear but discontinuous.
    \end{claim}

    Note that $T(P_n) = \left. nx^{n - 1} \right|_{x = 1} = n$ and that $\lVert P_n \rVert_\infty = 1$. This implies that
    $$ \frac{|T(P_n)|}{\lVert P_n \rVert} \xrightarrow{n \to \infty} \infty. $$
  \end{enumerate}
\end{enumerate}

\begin{thm}
  \normalfont Let $V$ be a normed vector space. A linear map $T : V \to \RR$ is continuous if and only if $T$ obeys
  \begin{equation} \label{eq:*} \tag{*}
    \sup \left\{ \frac{|T(\mathbf x)|}{\lVert \mathbf x \rVert} : \mathbf x \in V \right\} = K < \infty \Leftrightarrow |T(\mathbf x)| \leq K \lVert \mathbf x \rVert.
  \end{equation}
\end{thm}

\begin{proof}
  ``$\Leftarrow$'': {\bf Exercise:} If $T$ obeys \eqref{eq:*}, show that $T$ is Lipschitz and hence continuous.

  ``$\Rightarrow$'': Assume $T$ is linear and continuous on $V$, hence at $\mathbf 0$; i.e., given any $\epsilon > 0$ (choose $\epsilon = 1$), there exists $\delta > 0$ such that
  $$ \lVert \mathbf x \rVert = \lVert \mathbf x - \mathbf 0 \rVert < \delta \Rightarrow \left|T(\mathbf x) - \underbrace{T(\mathbf 0)}_{= 0}\right| < \epsilon = 1. $$

  \noindent {\bf Shown:}
  \begin{equation} \label{eq:**} \tag{**}
    \lVert \mathbf x \rVert < \delta \Rightarrow |T(\mathbf x)| < 1.
  \end{equation}
  Choose any $\mathbf y \in V, \mathbf y \neq \mathbf 0$. Set $\mathbf v = \frac{\delta}{2} \frac{\mathbf y}{\lVert \mathbf y \rVert}$. Then, $\lVert \mathbf v \rVert = 1 \cdot \frac{\delta}{2} = \frac{\delta}{2} < \delta$. Note that
  \begin{align*}
    \eqref{eq:**} &\Rightarrow |T(\mathbf v)| < 1 \\
    &\Rightarrow \left|T\left(\frac{\delta}{2} \frac{\mathbf y}{\lVert \mathbf y \rVert}\right)\right| < 1 \\
    &\Rightarrow \frac{\delta}{2} \frac{1}{\lVert \mathbf y \rVert} |T(\mathbf y)| < 1 \\
    &\Rightarrow |T(\mathbf y)| < \underbrace{\frac{2}{\delta}}_{= K} \lVert \mathbf y \rVert.
  \end{align*}
  This is \eqref{eq:*} with $K = \frac{2}{\delta}$.
\end{proof}

\subsection{Riesz Representation Theorem}

\begin{defn}
  \normalfont Let $V$ be a normed vector space. Then $V^* = \left\{ T : V \xrightarrow[\text{linear}]{\text{continuous}} \RR \right\}$ is called the {\bf dual} of $V$.
\end{defn}

\noindent {\bf Question:} What is $C[a, b]^*$? i.e., what does a continuous linear functional on $C[a, b]$ look like?

\begin{thm}[Riesz representation theorem]
  \normalfont $C[a, b]^* \cong BV[a, b]$.

  In other words, $L : C[a, b] \to \RR$ is a continuous linear functional if and only if there exists $\alpha \in BV[a, b]$ such that
  $$ L(f) = \int_a^b fd\alpha ~ \forall f \in C[a, b]. $$
\end{thm}

\begin{remark}
  \normalfont ~
  
  \begin{enumerate}
  \item Is $\alpha$ unique?
    
    {\bf Answer:} No. If $\alpha$ works, $\alpha + \text{constant}$ will absolutely work.

    In fact, there exists a choice of $\alpha \in BV[a, b]$ such that $\alpha$ is right continuous on $[a, b]$ and $\alpha(a) = 0$. Such a choice is unique.
  \item $\alpha$ is, in the language of measure, the \emph{distribution function} of a random variable; i.e., if $X$ is a random variable, $\alpha(x) = \mathbb P(X \leq x)$.
  \end{enumerate}
\end{remark}

\end{document}
