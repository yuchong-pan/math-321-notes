
\documentclass[letterpaper, reqno,11pt]{article}
\usepackage[margin=1.0in]{geometry}
\usepackage{color,latexsym,amsmath,amssymb}
\usepackage{fancyhdr}
\usepackage{amsthm}
\usepackage{mathtools}
\usepackage{tikz}
\usepackage{float}
\usepackage{centernot}
\usepackage{subcaption}
\usepackage{extarrows}
\usetikzlibrary{hobby}
\usetikzlibrary{shapes.multipart}
\usepackage{pgfplots}
\pgfplotsset{compat=1.7}
\usetikzlibrary{arrows.meta}
\usepackage{cancel}

\newcommand{\RR}{\mathbb{R}}
\newcommand{\CC}{\mathbb{C}}
\newcommand{\ZZ}{\mathbb{Z}}
\newcommand{\QQ}{\mathbb{Q}}
\newcommand{\NN}{\mathbb{N}}
\def\upint{\mathchoice%
  {\mkern13mu\overline{\vphantom{\intop}\mkern7mu}\mkern-20mu}%
  {\mkern7mu\overline{\vphantom{\intop}\mkern7mu}\mkern-14mu}%
  {\mkern7mu\overline{\vphantom{\intop}\mkern7mu}\mkern-14mu}%
  {\mkern7mu\overline{\vphantom{\intop}\mkern7mu}\mkern-14mu}%
  \int}
\def\lowint{\mkern3mu\underline{\vphantom{\intop}\mkern7mu}\mkern-10mu\int}
\DeclareMathOperator{\card}{card}
\DeclareMathOperator{\Binomial}{Binomial}
\DeclareMathOperator{\Span}{span}
\pagestyle{fancy}
\lhead{Math 321 Lecture 22}
\rhead{Yuchong Pan}
\begin{document}
\pagenumbering{arabic}
\title{Math 321 Lecture 22}
\author{Yuchong Pan}
\date{March 1, 2019}
\newtheorem{thm}{Theorem}
\newtheorem{defn}{Definition}
\newtheorem*{remark}{Remark}
\newtheorem{claim}{Claim}
\newtheorem{cor}{Corollary}
\newtheorem{lemma}{Lemma}
\newtheorem{prop}{Proposition}
\newtheorem{fact}{Fact}
\maketitle
%

\section{Fourier Series}

Let $f \in \mathcal R[-\pi, \pi]$. Recall: $\alpha_0 + \sum_{k = 1}^n (\alpha_k \cos kx + \beta_k \sin kx)$ is called the {\bf Fourier series} of $f$ if
\begin{equation} \label{eq:*} \tag{*}
  \left\{
  \begin{array}{lclr}
    \alpha_0 &=& \frac{1}{2\pi} \int_{-\pi}^\pi f(x) dx, & \\
    \alpha_k &=& \frac{1}{\pi} \int_{-\pi}^\pi f(x) \cos kx dx, & k \geq 1, \\
    \beta_k &=& \frac{1}{\pi} \int_{-\pi}^\pi f(x) \sin kx dx, & k \geq 1.
  \end{array}
  \right.
\end{equation}
As of now, we do not know if this series converges for \emph{any} $x \in [-\pi, \pi]$.

\begin{defn}
  \normalfont The $n^\text{th}$ {\bf partial Fourier series/sum} of $f$ is given by
  \[ (S_n f)(x) = \alpha_0 + \sum_{k = 1}^n (\alpha_k \cos kx + \beta_k \sin kx), \]
  where $\alpha_0, \alpha_k, \beta_k$ are as in \eqref{eq:*} for $1 \leq k \leq n$. Note that $S_n f$ is a well-defined trignometric polynomial for every $n \geq 1$.
\end{defn}

\noindent {\bf Question:} What do $S_n f$ represent, especially since $\{ S_n f : n \geq 1 \}$ need not converge to $f$ pointwise?

Consider the following space of functions in $[-\pi, \pi]$. Fix any $p \in [1, +\infty)$,
\[ L_*^p[-\pi, \pi] := \left\{ f : [-\pi, \pi] \xrightarrow[\text{$2\pi$-periodic}]{\text{continuous}} \CC \right\}, \]
equipped with the norm
\[ \lVert f \rVert_p = \left[\frac{1}{2\pi} \int_{-\pi}^\pi |f(x)|^p dx\right]^\frac{1}{p}, \qquad 1 \leq p < \infty. \]
For $p = \infty$,
\[ \lVert f \rVert_\infty \xlongequal{\text{def}} \sup \{ |f(x)| : x \in [-\pi, \pi] \}. \]
\fbox{{\bf Special case:} $p = 2$.}

\begin{thm}
  \normalfont Consider the following minimization problem: Take any $f \in L_*^2[-\pi, \pi]$.

  \medskip

  \noindent What is $\min\{ \lVert f - T \rVert : T \in \Span\{ 1, \cos kx, \sin kx : 1 \leq k \leq n \}$?

  \medskip

  \noindent When is the minimum attained?

  \medskip

  \noindent {\bf Answer:} $\min\{ \lVert f - T \rVert : T \in \Span\{ 1, \cos kx, \sin kx : 1 \leq k \leq n \}$ is attained if and only if $T = S_n f = \text{$n^\text{th}$ Fourier series}$.
\end{thm}

Note that
\[ \underbrace{\dim\left(L_*^2[-\pi, \pi]\right)}_{\{ 1, \sin kx, \cos kx : k \geq 1 \}} = \infty. \]

\begin{figure}[H]
  \centering
  \begin{tikzpicture}
    \draw (0, 0) -- (10, 0) node [right] {$V = \Span\{ 1, \sin kx, \cos kx : 1 \leq k \leq n \}$};
    \draw[red, thick] (5, 0) -- (5, 2.5);
    \draw (0.5, 0) -- (5, 2.5);
    \draw (1.5, 0) -- (5, 2.5);
    \draw (2.5, 0) -- (5, 2.5);
    \draw[fill=black] (5, 0) circle (2pt) node [below] {$T = S_n f$};
    \draw[fill=black] (2.5, 0) circle (2pt) node [below] {$\mathbf 0$};
    \draw[fill=black] (5, 2.5) circle (2pt) node[right] {$f \in L_*^2$};
    \node at (5, 4) {$L_*^2$};
    \draw[decoration={brace, raise=0.1cm, amplitude=5pt}, decorate] (5, 0) -- (5, 2.5);
  \end{tikzpicture}
\end{figure}

\begin{proof}
  We have
  \[ \lVert f - T \rVert_2^2 = \int_{-\pi}^\pi (f(x) - T(x))^2 dx = \int_{-\pi}^\pi (f(x))^2 dx - 2 \int_{-\pi}^\pi f(x) T(x) dx + \int_{-\pi}^\pi (T(x))^2 dx. \]
  Suppose $T(x) = a_0 + \sum_{k = 1}^n (a_k \cos kx + b_k \cos kx)$, $a_0, a_k, b_k \in \CC$. Then,
  \begin{align*}
    \lVert f - T \rVert_2^2 &= \int_{-\pi}^\pi (f(x))^2 dx - 2 \int_{-\pi}^\pi f(x) \left[a_0 + \sum_{k = 1}^n (a_k \cos kx + b_k \cos kx)\right] dx + \int_{-\pi}^\pi (T(x))^2 dx \\
    &= \int_{-\pi}^\pi (f(x))^2 dx - 2 \left[a_0 \underbrace{\int_{-\pi}^\pi f(x) dx}_{= 2\pi \alpha_0} + \sum_{k = 1}^n \left(a_k \underbrace{\int_{-\pi}^\pi f(x) \cos kx dx}_{= \pi \alpha_k} + b_k \underbrace{\int_{-\pi}^\pi f(x) \sin kx dx}_{= \pi \beta_k}\right)\right] \\
    & \quad + \int_{-\pi}^\pi (T(x))^2 dx \\
    &= \int_{-\pi}^\pi (f(x))^2 dx - 2\pi \left[2\alpha_0 a_0 + \sum_{k = 1}^n (a_k \alpha_k + b_k \beta_k)\right] + a_0^2 \cdot 2\pi + \pi \sum_{k = 1}^n \left(a_k^2 + b_k^2\right) \\
    &= \int_{-\pi}^\pi (f(x))^2 dx + 2\pi \underbrace{\left[a_0^2 - 2\alpha_0 a_0\right]}_{= (a_0 - \alpha_0)^2 - \alpha_0^2} + \pi \sum_{k = 1}^n \underbrace{\left(a_k^2 - 2\alpha_k a_k\right)}_{= (a_k - \alpha_k)^2 - \alpha_k^2} + \pi \sum_{k = 1}^n \underbrace{\left(b_k^2 - 2\beta_k b_k\right)}_{= (b_k - \beta_k)^2 - \beta_k^2} \\
    &= \underbrace{\int_{-\pi}^\pi (f(x))^2 dx - \pi\left[2\alpha_0^2 + \sum_{k = 1}^n \left(\alpha_k^2 + \beta_k^2\right)\right]}_\text{independent of $a_k, b_k$ (dependent on $f$)} + \underbrace{\pi \left[(2(a_0 - \alpha_0)^2 + \sum_{k = 1}^n \left((a_k - \alpha_k)^2 + (b_k - \beta_k)^2\right)\right]}_{\geq 0} \\
    &\geq \underbrace{\int_{-\pi}^\pi (f(x))^2 dx - \pi\left[2\alpha_0^2 + \sum_{k = 1}^n \left(\alpha_k^2 + \beta_k^2\right)\right]}_\text{minimum value, attained when $a_0 = \alpha_0, a_k = \alpha_k, b_k = \beta_k$}.
  \end{align*}
\end{proof}

\noindent {\bf Exercise:} $\lVert f - S_n f \rVert_2^2 = \lVert f \rVert_2^2 - \lVert S_n f \rVert_2^2$.

\end{document}
