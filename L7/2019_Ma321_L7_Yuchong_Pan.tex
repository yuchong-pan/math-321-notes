
\documentclass[letterpaper, reqno,11pt]{article}
\usepackage[margin=1.0in]{geometry}
\usepackage{color,latexsym,amsmath,amssymb}
\usepackage{fancyhdr}
\usepackage{amsthm}
\usepackage{mathtools}
\usepackage{tikz}
\usepackage{float}
\usepackage{centernot}
\usepackage{subcaption}
\usepackage{extarrows}
\usetikzlibrary{hobby}
\usetikzlibrary{shapes.multipart}
\usepackage{pgfplots}
\pgfplotsset{compat=1.7}

\newcommand{\RR}{\mathbb{R}}
\newcommand{\CC}{\mathbb{C}}
\newcommand{\ZZ}{\mathbb{Z}}
\newcommand{\QQ}{\mathbb{Q}}
\newcommand{\NN}{\mathbb{N}}
\DeclareMathOperator{\card}{card}
\DeclareMathOperator{\Binomial}{Binomial}
\pagestyle{fancy}
\lhead{Math 321 Lecture 7}
\rhead{Yuchong Pan}
\begin{document}
\pagenumbering{arabic}
\title{Math 321 Lecture 7}
\author{Yuchong Pan}
\date{January 16, 2019}
\newtheorem{thm}{Theorem}
\newtheorem{defn}{Definition}
\newtheorem*{remark}{Remark}
\newtheorem{claim}{Claim}
\newtheorem{cor}{Corollary}
\newtheorem{lemma}{Lemma}
\maketitle
%

\section{The Space $C(X)$ (Cont'd)}

Let $(X, d)$ be any metric space. Then $C(X)$ is the space of real-valued or complex-valued continuous functions on $X$. $C(X)$ is always a vector space on $\RR$ or $\CC$. If $X$ is {\bf compact}, then $C(X)$ is also a metric space, with respect to the uniform norm: $\lVert f - g \rVert_\infty = \sup_{x \in X} |f(x) - g(x)|$.

~

\noindent {\bf Questions:}
\begin{enumerate}
\item Suppose $X$ is compact, so that $C(X)$ is a metric space. What are the compact subsets of $C(X)$?
\item {\bf Know:}
  \begin{align*}
    \text{uniform convergence} &\Rightarrow \text{pointwise convergence} \\
    \text{pointwise convergence} &\centernot\Rightarrow \text{uniform convergence}
  \end{align*}
  pointwise convergence $+$ \fbox{equicontinuity} $\Rightarrow$ uniform convergence.
\end{enumerate}

\subsection{Arzela-Ascoli Theorem}

\begin{thm}[Arzela-Ascoli] \label{thm:1}
  \normalfont Let $(X, d)$ be a {\bf compact} metric space. Let $\mathcal F \subseteq C(X)$. Then $\mathcal F$ is compact if and only if $\mathcal F$ is closed, uniform bounded ($\exists M > 0 \text{ s.t. } \lVert F \rVert < M ~ \forall F \in \mathcal F$) and equicontinuous.
\end{thm}

Recall that $\mathcal G \subseteq C(X)$ where $X$ is compact is {\bf equicontinuous} if for all $\epsilon > 0$, there exists $\delta = \delta(\epsilon)$ such that for all $x, y \in X$ with $d(x, y) < \delta$, one has $|f(x) - f(y)| < \epsilon$ for all $f \in \mathcal G$.

~

\noindent {\bf Examples:}
\begin{enumerate}
\item $\mathcal G = \{ \text{ all constant functions } \}$ is equicontinuous but not uniformly bounded; AA $\Rightarrow$ $\mathcal G$ is not compact.
\item $\mathcal G = \left\{ f_n(x) = x^n ; x \in [0, 1] \right\}$ is uniformly bounded by $1$, but not equicontinuous (Exercise).

  Check that $\delta_n$ has to go to zero as $n \to \infty$ (near 1).
\item $\mathcal G = \{ \text{ all constant functions $f(x) = k, k \in [0, 1]$} \}$ is equicontinuous and uniformly bounded.
\item Any finite collection of continuous functions on a compact metric space is both equicontinuous and uniformly bounded.
\item $\mathcal G = \left\{ x^n ; x \in \left[0, \frac{1}{2}\right] \right\} \subseteq C\left[0, \frac{1}{2}\right]$ and $\mathcal G = \left\{ \frac{x}{n} ; x \in [0, 1] \right\}$ are equicontinuous and uniformly bounded.
\item The space of polynomials on $[0, 1]$ is neither equicontinuous nor uniformly bounded.
\end{enumerate}

\noindent {\bf Exercise:} Show that if $f_n, f \in C(X)$ where $X$ is compact, $f_n \to f$ uniformly on $X$, then $\mathcal F = \{ f_n : n \geq 1 \} \cup \{ f \}$ is equicontinuous and uniformly bounded.

\begin{proof}[Proof of Theorem \ref{thm:1}]
  \renewcommand{\qedsymbol}{}
  \noindent {\bf Step 1:} Assume $\mathcal F$ is compact. Show that $\mathcal F$ is closed, uniformly bounded and equicontinuous.
  \begin{itemize}
  \item $\mathcal F$ is closed because compactness implies closedness in any metric space.
  \item $\mathcal F$ is uniformly bounded: $\mathcal F$ compact $\Rightarrow$ $\mathcal F$ $\underbrace{\text{totally bounded}}_\text{i.e., $\forall \epsilon > 0, \exists f_1, f_2, \ldots, f_M \in C(X) \text{ s.t. } \mathcal F \subseteq \bigcup_{j = 1}^M B(f_j ; \epsilon)$}$.
    
    Choose $\epsilon = 1$. Then
    $$ \mathcal F \subseteq \bigcup_{j = 1}^M \underbrace{\{ f \in C(X) : \lVert f - f_j \rVert_\infty < 1 \}}_{= B(f_j; 1)}. $$

    Thus, for any $f \in \mathcal F$, there exists $f_j$ such that $\lVert f - f_j \rVert_\infty < 1$; i.e.,
    $$ |f(x)| \leq |f(x) - f_j(x)| + |f_j(x)| \leq \lVert f - f_j \rVert_\infty + \lVert f_j \rVert < 1 + N, $$
    where $N = \max \{ \lVert f_j \rVert : 1 \leq j \leq M \} < \infty$.
  \item $\mathcal F$ is equicontinuous: Fix any $\epsilon > 0$.

    Since $\mathcal F$ is totally bounded, identify $f_1, \ldots, f_M \in C(X)$ such that
    \begin{equation} \label{eq:star} \tag{*}
      \mathcal F \subseteq \bigcup_{j = 1}^M B\left(f_j ; \frac{\epsilon}{3}\right).
    \end{equation}
    For every $1 \leq j \leq M$, there exists $\delta_j = \delta_j(\epsilon) > 0$ such that
    \begin{equation} \label{eq:star-star} \tag{**}
      d(x, y) < \delta_j \xRightarrow{\text{$f$ is continuous}} |f_j(x) - f_j(y)|< \frac{\epsilon}{3} \quad \forall x, y \in X.
    \end{equation}
    Set $\delta = \min(\delta_1, \delta_2, \ldots, \delta_M) > 0$. We need to show that this $\delta$ works for all $f \in \mathcal F$:
    $$ |f(x) - f(y)| \leq |f(x) - f_j(x)| + |f_j(x) - f_j(y)| + |f_j(y) - f(y)|, $$
    where $1 \leq j \leq M$ is chosen so that $\lVert f - f_j \rVert < \frac{\epsilon}{3}$ (by \eqref{eq:star}). Thus,
    \begin{align*}
      |f(x) - f(y)| &< 2\lVert f - f_j \rVert + \frac{\epsilon}{3} \qquad \text{by \eqref{eq:star-star}} \\
      &< 2 \cdot \frac{\epsilon}{3} + \frac{\epsilon}{3} = \epsilon.
    \end{align*}
    whenever $d(x, y) < \delta$.
  \end{itemize}

  ~

  (Proof unfinished.)
\end{proof}

\end{document}
