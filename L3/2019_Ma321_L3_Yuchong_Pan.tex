
\documentclass[letterpaper, reqno,11pt]{article}
\usepackage[margin=1.0in]{geometry}
\usepackage{color,latexsym,amsmath,amssymb}
\usepackage{fancyhdr}
\usepackage{amsthm}
\usepackage{mathtools}
\usepackage{tikz}
\usepackage{float}
\usepackage{centernot}
\usepackage{subcaption}
\usepackage{extarrows}

\newcommand{\RR}{\mathbb{R}}
\newcommand{\CC}{\mathbb{C}}
\newcommand{\ZZ}{\mathbb{Z}}
\newcommand{\QQ}{\mathbb{Q}}
\newcommand{\NN}{\mathbb{N}}
\pagestyle{fancy}
\lhead{Math 321 Lecture 3}
\rhead{Yuchong Pan}
\begin{document}
\pagenumbering{arabic}
\title{Math 321 Lecture 3}
\author{Yuchong Pan}
\date{January 7, 2019}
\newtheorem{thm}{Theorem}
\newtheorem{defn}{Definition}
\newtheorem{exs}{Exercise}
\newtheorem{remark}{Remark}
\newtheorem{claim}{Claim}
\maketitle
%

\section{Theorem and Consequences}

\subsection{Theorem and Proof}

\begin{thm} \label{thm:1}
  \normalfont Let $(X, d)$ and $(Y, \rho)$ be metric spaces. Assume that $f_n : X \to Y$ is continuous for $n \geq 1$ and that $f_n \to f$ uniformly on $X$. Then $f$ is continuous on $X$.
\end{thm}

\begin{proof}
  Fix $x_0 \in X$ and $\epsilon > 0$. We need to find $\delta = \delta(\epsilon, x_0) > 0$ such that
  $$ \rho(f_(x), f(x_0)) < \epsilon $$
  whenever $d(x, x_0) < \delta$.

  {\bf Given:}
  \begin{align*}
    & \quad f_n \xrightarrow{n \to \infty} f \text{\bf ~uniformly on $X$} \\
    \Leftrightarrow & \quad \sup_{x \in X} \rho(f_n(x), f(x)) \xrightarrow{n \to \infty} 0 \\
    \Leftrightarrow & \quad \text{given any $\epsilon > 0$, there exists $N \geq 1$ such that} \\
    & \quad \sup_{x \in X} \rho(f_n(x), f(x)) < \frac{\epsilon}{3} \qquad \text{whenever $n \geq N$}. \tag{*} \label{eq:star}
  \end{align*}

  Thus, by the triangular inequality applied twice,
  $$ \rho(f(x), f(x_0)) \leq \underbrace{\rho(f(x), f_N(x))}_\text{(1)} + \underbrace{\rho(f_N(x), f_N(x_0))}_\text{(2)} + \underbrace{\rho(f_N(x_0), f(x_0))}_\text{(3)}. $$
  Both (1) and (3) are buonded above by $\sup_{y \in X} \rho(f_N(y), f(y))$, which is $< \frac{\epsilon}{3}$ by (\ref{eq:star}).
  Therefore,
  $$ \rho(f(x), f(x_0)) < \frac{2\epsilon}{3} + \underbrace{\rho(f_N(x), f_N(x_0))}_\text{(2)} < \frac{2\epsilon}{3} + \frac{\epsilon}{3}, $$
  if we choose $\delta = \delta_N(\epsilon, x_0) > 0$ such that $\rho(f_N(x), f_N(x_0)) < \frac{\epsilon}{3}$ whenever $d(x, x_0) < \delta_N$, by the $\epsilon$-$\delta$ definition of the continuity of $f_N$ at $x_0 \in X$.
\end{proof}

\noindent {\bf Example:} $k_n(x) = x^n$ is continuous on $[0, 1]$. Last time, we showed that $k_n \xrightarrow{\text{pointwise}} k$, where
$$ k(x) = \left\{
\begin{array}{ll}
  0, & x \neq 1, \\
  1, & x = 1,
\end{array}
\right. $$
which is discontinuous.

\subsection{Consequences}

\begin{enumerate}
\item Suppose $X = [a, b]$ and $Y = \RR \text{ or } \CC$. Then Theorem \ref{thm:1} implies that $C[a, b]$ is closed under {\bf uniform limits}.

  That is, any sequence in $C[a, b]$, if it converges uniformly, admits the limit in $C[a, b]$; i.e., if $\{ f_n \} \subseteq C[a, b]$ and $f_n \xrightarrow{n \to \infty} f$ uniformly on $C[a, b]$, then $f \in C[a, b]$.

  Note that $C[a, b]$ is closed in $\mathcal B[a, b] = \left\{ f : [a, b] \xrightarrow{\text{bounded}} \RR \text{ or } \CC \right\}$, a metric space with metric $d(f, g) = \sup_{x \in a[a, b]} |f(x) - g(x)|$.
\item $C[a, b]$ is {\bf complete}.

  That is, every Cauchy sequence $\{ f_n \} \subseteq C[a, b]$ (i.e., $\sup_{x \in [a, b]} |f_n(x) - f_m(x)| \xrightarrow{n, m \to \infty} 0$) converges.

  \begin{enumerate}
  \item[{\bf Step 1:}] Find a candidate for $\lim_{n \to \infty} f_n$.

    {\bf Know:} $\lVert f_n - f_m \rVert_\infty = \sup_{x \in [a, b]} |f_n(x) - f_m(x)| \xrightarrow{n, m \to \infty} 0$.

    Fix any $x \in [a, b]$. Then $\{ f_n(x) \}$ is a sequence of real (or complex) numbers, of $|f_n(x) - f_m(x)| \leq \lVert f_n - f_m \rVert_\infty \xrightarrow{n, m \to \infty} 0$.

    Since $\RR$ and $\CC$ are complete, $\lim_{n \to \infty} f_n(x)$ exists for all $x \in [a, b]$. Set $f(x) = \lim_{n \to \infty} f_n(x)$. Note that $f$ is pointwise convergent so far.
  \item[{\bf Step 2:}] We need to show:
    \begin{enumerate}
    \item $f_n \xrightarrow{n \to \infty} f$ uniformly on $[a, b]$, and
    \item $f \in C[a, b]$.
    \end{enumerate}
    \begin{enumerate}
    \item
      \begin{align*}
        \sup_{x \in [a, b]} |f_n(x) - f(x)| &= \sup_{x \in [a, b]} \left|f_n(x) - \lim_{m \to \infty} f_m(x)\right| \\
        &= \sup_{x \in [a, b]} \left|\lim_{m \to \infty} (f_n(x) - f_m(x))\right| \\
        &= \sup_{x \in [a, b]} \lim_{m \to \infty} |f_n(x) - f_m(x)| \\
        &\leq \sup_{x \in [a, b]} \underbrace{\lim_{m \to \infty} \lVert f_n - f_m \rVert_\infty}_\text{does not depend on $x$} \\
        &= \lim_{m \to \infty} \lVert f_n - f_m \rVert_\infty \\
        &< \epsilon,
      \end{align*}
      if $n \geq N_\epsilon$ (using the fact that $\{ f_n \} \subseteq C[a, b]$ is Cauchy).
    \item $f \in C[a, b]$ by Theorem \ref{thm:1}.
    \end{enumerate}
  \end{enumerate}
\item {\bf Weierstrass $M$-test}:

  Let $\{ g_n \} \subseteq \mathcal B(X) = \left\{ f : X \xrightarrow{\text{bounded}} \RR \text{ or } \CC \right\}$ be such that $\sum_{n = 1}^\infty \lVert g_n \rVert_\infty < \infty$.

  Then $g = \sum_{n = 1}^\infty g_n$ converges uniformly on $X$; i.e., $\left\{ S_n(x) = \sum_{n = 1}^N g_n(x) : N \geq 1 \right\}$ converges uniformly on $X$. In addition, $g \in \mathcal B(X)$.

  Moreover, if $\{ g_n \} \subseteq C(X)$, then $g \in C(X)$.
\end{enumerate}

\end{document}
