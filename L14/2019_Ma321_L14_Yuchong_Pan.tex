
\documentclass[letterpaper, reqno,11pt]{article}
\usepackage[margin=1.0in]{geometry}
\usepackage{color,latexsym,amsmath,amssymb}
\usepackage{fancyhdr}
\usepackage{amsthm}
\usepackage{mathtools}
\usepackage{tikz}
\usepackage{float}
\usepackage{centernot}
\usepackage{subcaption}
\usepackage{extarrows}
\usetikzlibrary{hobby}
\usetikzlibrary{shapes.multipart}
\usepackage{pgfplots}
\pgfplotsset{compat=1.7}

\newcommand{\RR}{\mathbb{R}}
\newcommand{\CC}{\mathbb{C}}
\newcommand{\ZZ}{\mathbb{Z}}
\newcommand{\QQ}{\mathbb{Q}}
\newcommand{\NN}{\mathbb{N}}
\def\upint{\mathchoice%
  {\mkern13mu\overline{\vphantom{\intop}\mkern7mu}\mkern-20mu}%
  {\mkern7mu\overline{\vphantom{\intop}\mkern7mu}\mkern-14mu}%
  {\mkern7mu\overline{\vphantom{\intop}\mkern7mu}\mkern-14mu}%
  {\mkern7mu\overline{\vphantom{\intop}\mkern7mu}\mkern-14mu}%
  \int}
\def\lowint{\mkern3mu\underline{\vphantom{\intop}\mkern7mu}\mkern-10mu\int}
\DeclareMathOperator{\card}{card}
\DeclareMathOperator{\Binomial}{Binomial}
\pagestyle{fancy}
\lhead{Math 321 Lecture 14}
\rhead{Yuchong Pan}
\begin{document}
\pagenumbering{arabic}
\title{Math 321 Lecture 14}
\author{Yuchong Pan}
\date{February 4, 2019}
\newtheorem{thm}{Theorem}
\newtheorem{defn}{Definition}
\newtheorem*{remark}{Remark}
\newtheorem{claim}{Claim}
\newtheorem{cor}{Corollary}
\newtheorem{lemma}{Lemma}
\newtheorem{prop}{Proposition}
\maketitle
%

\section{Riemann-Stieltjes Integration}

\subsection{Definitions}

\begin{defn}
  \normalfont Let
  \begin{align*}
    & [a, b] \overset{\text{\small compact}}{\subseteq} \RR, \\
    & \alpha : [a, b] \to \RR \text{ be } \underbrace{\text{non-constant}}_{\exists x, y \in [a, b], \alpha(x) \neq \alpha(y)} \text{, } \underbrace{\text{non-decreasing}}_{\text{if $x < y$ then $\alpha(x) \leq \alpha(y)$}} \text{, called the {\bf integrator}}, \\
    & f : [a, b] \to \RR \text{ be bounded, called the {\bf integrand}}.
  \end{align*}
  Given any partition $P = \{ a = x_0 < x_1 < \ldots < x_n = b \}$ of $[a, b]$, define
  $$ \left.
  \begin{array}{lcl}
    L_\alpha(f, P) &=& \sum_{i = 1}^n m_i \Delta \alpha_i, \\
    U_\alpha(f, P) &=& \sum_{i = 1}^n M_i \Delta \alpha_i,
  \end{array}
  \right\}
  \text{ where }
  \begin{array}{lcl}
    m_i &=& \inf \{ f(x) : x_{i - 1} \leq x \leq x_i \}, \\
    M_i &=& \sup \{ f(x) : x_{i - 1} \leq x \leq x_i \}, \\
    \Delta \alpha_i &=& \alpha(x_i) - \alpha(x_{i - 1}).
  \end{array}
  $$
\end{defn}

\noindent {\bf Exercises:}
\begin{enumerate}
\item $L_\alpha(f, P) \leq U_\alpha(f, P)$.
\item If $Q$ refines $P$, i.e. $P \subsetneq Q$, then
  \begin{equation} \label{eq:*} \tag{*}
    L_\alpha(f, P) \leq L_\alpha(f, Q) \leq U_\alpha(f, Q) \leq U_\alpha(f, P).
  \end{equation}
\end{enumerate}

\begin{defn}
  \normalfont
  \begin{align*}
    & \text{{\bf Lower RS integral} of $f$ with respect to $\alpha$ on $[a, b]$} = \lowint_a^b fd\alpha = \lowint_a^b f(x) d\alpha(x) \overset{\text{def}}{=} \sup_\text{partition $P$ of $[a, b]$} L_\alpha(f, P), \\
    & \text{{\bf Upper RS integral} of $f$ with respect to $\alpha$ on $[a, b]$} = \upint_a^b fd\alpha = \upint_a^b f(x) d\alpha(x) \overset{\text{def}}{=} \inf_\text{partition $P$ of $[a, b]$} U_\alpha(f, P).
  \end{align*}
\end{defn}

\begin{remark}
  \normalfont $$ \eqref{eq:*} \Rightarrow \lowint_a^b fd\alpha \leq \upint_a^b fd\alpha \text{ for any $f$}. $$
\end{remark}

\begin{defn}
  \normalfont We say that $f$ is {\bf Riemann-Stieltjes integrable} with respect to $\alpha$, written $f \in \mathcal R_\alpha[a, b]$, if
  $$ \lowint_a^b fd\alpha = \upint_a^b fd\alpha. $$
  In this case,
  $$ \lowint_a^b fd\alpha = \upint_a^b fd\alpha \xlongequal{\text{notation}} \int_a^b fd\alpha = \text{the {\bf RS integral} of $f$ with respect to $\alpha$ on $[a, b]$}. $$
\end{defn}

\subsection{Examples}

\begin{enumerate}
\item $\alpha(x) = x$; the standard Riemann integration.

  \noindent {\bf Applications}:
  \begin{enumerate}
  \item Represents signed area under a curve.
  \item
    $$ \frac{d}{dx} \int_a^x f(t) dt = f(x). $$
    Indefinite Riemann integrals are ``antiderivatives'', i.e., provide inverses to differentiation.
  \item Errors in approximation of a function by its linear expansion involve a Riemann integral.
  \end{enumerate}
\item
  \begin{align*}
    & [a, b] = [0, 1], \\
    & \alpha(x) = \left\{
    \begin{array}{ll}
      0, & \text{if $0 \leq x < \frac{1}{2}$}, \\
      1, & \text{if $\frac{1}{2} \leq x \leq 1$}.
    \end{array}
    \right.
  \end{align*}

  \noindent Is {\bf every continuous $f$} on $[0, 1]$ in $\mathcal R_\alpha[0, 1]$ for this $\alpha$?

  \begin{figure}[H]
    \centering
    \begin{tikzpicture}
      \begin{axis}[
          xmin=0,
          xmax=1.2,
          ymin=0,
          ymax=1.2,
          xtick={0.1, 0.2, 0.3, 0.4, 0.5, 0.6, 0.7, 0.8, 0.9, 1},
          xticklabels={, , , , $\frac{1}{2}$, , , , , $1$},
          ymajorticks=false,
          axis x line=bottom,
          axis y line=left,
          x = 3cm,
          y = 3cm,
        ]
        \addplot[thick, red, samples at={0, 0.05, ..., 0.45, 0.5}](x,0);
        \addplot[thick, red, samples at={0.5, 0.55, ..., 0.95, 1}](x,1) node [right] {$\alpha$};
        \draw[dashed] (axis cs:0, 1) -- (axis cs:0.5, 1);
        \draw[dashed] (axis cs:1, 0) -- (axis cs:1, 1);
      \end{axis}
    \end{tikzpicture}
  \end{figure}

  $$ \Delta \alpha_i = \left\{
  \begin{array}{ll}
    0, & \text{if $\frac{1}{2} \not \in [x_{i - 1}, x_i]$}, \\
    1, & \text{if $\frac{1}{2} \in [x_{i - 1}, x_i]$}.
  \end{array}
  \right. $$
  \noindent {\bf Note:} There exist at least one and at most two choices of $i$ for which $\Delta \alpha_i$ can be $1$.

  \begin{figure}[H]
    \centering
    \begin{tikzpicture}[
        declare function={f(\x)=1000*\x*(\x+0.1)*(\x-0.1)*(\x-0.45)*(\x-0.7)*(\x-0.7)*(\x-1)*(\x-1.2)+0.5;},
      ]
      \begin{axis}[
          xmin=0,
          xmax=1.2,
          ymin=0,
          ymax=1.2,
          xtick={0, 0.4, 0.5, 0.6, 1},
          xticklabels={$0$, $x_{i - 1}$, $\frac{1}{2}$, $x_i$, $1$},
          ymajorticks=false,
          axis x line=bottom,
          axis y line=left,
        ]
        \addplot[thick, red, samples at={0, 0.05, ..., 0.95, 1}, smooth](x,{f(x)});
        \draw[dashed] (axis cs:0.5428, 0) -- (axis cs:0.5428, {f(0.5428)});
        \draw[dashed] (axis cs:0.4, 0) -- (axis cs:0.4, {f(0.4)});
        \draw[dashed] (axis cs:0.6, 0) -- (axis cs:0.6, {f(0.6)});
        \draw[dashed] (axis cs:0, {f(0.4)}) -- (axis cs:0.4, {f(0.4)}) node [above] {$m_i$};
        \draw[dashed] (axis cs:0, {f(0.5428)}) -- (axis cs:0.5428, {f(0.5428)}) node [above] {$M_i$};
      \end{axis}
    \end{tikzpicture}
  \end{figure}

  Recall $f$ is continuous on $[0, 1]$, hence uniformly continuous. Therefore, given any $\epsilon > 0$, there exists $\delta > 0$ such that $|x - y| < \delta \Rightarrow |f(x) - f(y)| < \frac{\epsilon}{2}$.
  $$ U_\alpha(f, P) - L_\alpha(f, P) = \underbrace{\sum_{i = 1}^n \overbrace{(M_i - m_i)}^{< \frac{\epsilon}{2}} \Delta \alpha_i}_\text{at most $2$ nonzero terms where $\Delta \alpha_i = 1$} < \frac{\epsilon}{2} + \frac{\epsilon}{2} = \epsilon \Rightarrow f \in \mathcal R_\alpha[0, 1], $$
  {\bf provided} $\Delta x_i = x_i - x_{i - 1} < \delta$ for all $i$. Choose any $x, y \in [x_{i - 1}, x_i]$. Since $\Delta x_i < \delta$, then $|f(x) - f(y)| < \frac{\epsilon}{2}$. This implies that $M_i - m_i < \frac{\epsilon}{2}$.

  \begin{claim}
    \normalfont
    $$ \int_0^1 f\underbrace{d\alpha}_\text{Dirac delta at $x = \frac{1}{2}$} = f\left(\frac{1}{2}\right). $$
  \end{claim}

  \begin{proof}
    Fix $\epsilon > 0$. It suffices to show that
    $$ \left|U_\alpha(f, P) - f\left(\frac{1}{2}\right)\right| < \epsilon \text{ for sufficiently fine partitions $P$}. $$
    We have
    \begin{align*}
      U_\alpha(f, P) - f\left(\frac{1}{2}\right) &= \sum_{i = 1}^n M_i \Delta \alpha_i - f\left(\frac{1}{2}\right) \\
      &= \underbrace{M_i^* \overbrace{\Delta \alpha_{i^*}}^{= 1}}_\text{where $i^*$ is the unique index where $\Delta \alpha_{i^*} = 1$} - f\left(\frac{1}{2}\right) \\
      &= \sup \{ f(x) : x \in [x_{i^* - 1}, x_{i^*}] \} - f\left(\frac{1}{2}\right) \\
      &= \sup \left\{ \underbrace{f(x) - f\left(\frac{1}{2}\right)}_{< \epsilon} : x \in [x_{i^* - 1}, x_{i^*}] \right\},
    \end{align*}
    if $x_{i^*} - x_{i^* - 1} < \delta$ by the continuity of $f$.
  \end{proof}

  \noindent {\bf Exercise:} Try this for
  $$ \alpha(x) = \left\{
  \begin{array}{ll}
    0, & \text{if $x \in \left[0, \frac{1}{2}\right)$}, \\
      \frac{1}{2}, & \text{if $x = \frac{1}{2}$}, \\
      1, & \text{if $x \in \left(\frac{1}{2}, 1\right]$}.
  \end{array}
  \right. $$

  \noindent {\bf Check that:}
  $$ f \in \mathcal R_\alpha[0, 1] \Leftrightarrow \text{$f$ is continuous at $x = \frac{1}{2}$}. $$
\end{enumerate}

\end{document}
