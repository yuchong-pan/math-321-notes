
\documentclass[letterpaper, reqno,11pt]{article}
\usepackage[margin=1.0in]{geometry}
\usepackage{color,latexsym,amsmath,amssymb}
\usepackage{fancyhdr}
\usepackage{amsthm}
\usepackage{mathtools}
\usepackage{tikz}
\usepackage{float}
\usepackage{centernot}
\usepackage{subcaption}
\usepackage{extarrows}

\newcommand{\RR}{\mathbb{R}}
\newcommand{\CC}{\mathbb{C}}
\newcommand{\ZZ}{\mathbb{Z}}
\newcommand{\QQ}{\mathbb{Q}}
\newcommand{\NN}{\mathbb{N}}
\pagestyle{fancy}
\lhead{Math 321 Lecture 4}
\rhead{Yuchong Pan}
\begin{document}
\pagenumbering{arabic}
\title{Math 321 Lecture 4}
\author{Yuchong Pan}
\date{January 9, 2019}
\newtheorem{thm}{Theorem}
\newtheorem{defn}{Definition}
\newtheorem{exs}{Exercise}
\newtheorem{remark}{Remark}
\newtheorem{claim}{Claim}
\maketitle
%

\section{Equicontinuous Families of Functions}

In Theorem 3.6, we saw that every bounded sequence of {\bf complex numbers} admits a convergent subsequence.

~

\noindent {\bf Question:} What about bounded sequences of {\bf functions}?

\subsection{Pointwise and Uniform Boundedness}

Let $\{ f_n \}$ be a sequence of functions defined on a set $E$.

\begin{defn}
  \normalfont We say that $\{ f_n \}$ is {\bf pointwise bounded} on $E$ if the sequence $\{ f_n(x) \}$ is bounded for every $x \in E$; i.e., there exists a finite-valued function $\phi$ defined on $E$ such that
  $$ |f_n(x)| < \phi(x) \qquad (x \in E, n = 1, 2, 3, \ldots). $$
\end{defn}

\begin{remark}
  \normalfont The bound $\phi(x)$ depends on $x$.
\end{remark}

\begin{defn}
  \normalfont We say that $\{ f_n \}$ is {\bf uniformly bounded} on $E$ if there exists a number $M$ such that
  $$ |f_n(x)| < M \qquad (x \in E, n = 1, 2, 3, \ldots). $$
\end{defn}

\begin{remark}
  \normalfont The bound $M$ is independent of $x$.
\end{remark}

\subsection{Examples}

\begin{enumerate}
\item $f_n(x) = \sin(nx), x \in E = [0, 2 \pi], n = 1, 2, 3, \ldots$.
  \begin{enumerate}
  \item $|f_n(x)| = |\sin(nx)| \leq 1 < 2$. Thus, $\{ f_n \}$ is uniformly bounded.
  \item We claim that $\{ f_n \}$ admits no subsequence $\{ f_{n_k} \}$ such that $\{ f_{n_k}(x) \}$ converges for every $x \in E = [0, 2\pi]$.
    \begin{proof}
      Suppose not; i.e., there exists a sequence $\{ n_k \}_{k = 1,2 , 3, \ldots}$ such that $\{ \sin(n_k x) \}$ converges for all $x \in E = [0, 2\pi]$.

      By the Cauchy criterion (for convergence),
      $$ \lim_{k \to \infty} (\sin(n_k x) - \sin(n_{k + 1} x)) = 0, \qquad x \in [0, 2\pi]. $$
      Hence,
      $$ \lim_{k \to \infty} (\sin(n_k x) - \sin(n_{k + 1} x))^2 = \left(\lim_{k \to \infty} (\sin(n_k x) - \sin(n_{k + 1} x)\right)^2 = 0^2 = 0. $$
      
      Lebesgue's Theorem states that if $f_n \to f$ pointwise as $n \to \infty$ and $|f_n(x)| < M$ for some $M \in \RR$, then $\lim_{n \to \infty} f_a^b f_n(x) dx = \int_a^b \lim_{n \to \infty} f_n(x) dx = \int_a^b f(x) dx$. Note that
      \begin{enumerate}
      \item $(\sin(n_k x) - \sin(n_{k + 1} x))^2 \to 0$ as $k \to \infty$;
      \item $(\sin(n_k x) - \sin(n_{k + 1} x))^2 \leq (1 + 1)^2 = 4 < 5$.
      \end{enumerate}
      By Lebesgue's Theorem,
      \begin{equation}\label{eq:1}
        \begin{split}
          & \quad \lim_{k \to \infty} \int_0^{2\pi} (\sin(n_k x) - \sin(n_{k + 1} x))^2 dx \\
          &= \int_0^{2\pi} \lim_{k \to \infty} (\sin(n_k x) - \sin(n_{k + 1} x))^2 dx \\
          &= \int_0^{2\pi} 0 dx \\
          &= 0.
        \end{split}
      \end{equation}

      On the other hand,
      \begin{equation*}
        \begin{split}
          & \quad \int_0^{2\pi} (\sin(n_k x) - \sin(n_{k + 1} x))^2 dx \\
          &= \int_0^{2\pi} \left(\sin^2(n_k x) + \sin^2(n_{k + 1} x) - 2 \sin(n_k x) \sin(n_{k + 1} x)\right) dx \\
          &= \int_0^{2\pi} \left(\frac{1 - \cos(2n_k x)}{2} + \frac{1 - \cos(2n_{k + 1} x)}{2} - (\cos((n_k - n_{k + 1}) x) - \cos((n_k + n_{k + 1}) x))\right) dx \\
          &= \int_0^{2\pi} 1 dx - \frac{1}{2} \int_0^{2\pi} \cos(2n_k x) dx - \frac{1}{2} \int_0^{2\pi} \cos(2n_{k + 1} x) dx \\
          & \qquad - \int_0^{2\pi} \cos((n_k - n_{k + 1}) x) dx - \int_0^{2\pi} \cos((n_k + n_{k + 1}) x) dx
        \end{split}
      \end{equation*}
      Note that $\int_0^{2\pi} \cos(mx) dx = \left.\frac{\sin(mx)}{m}\right|_0^{2\pi} = 0$ for $m \in \ZZ, m \neq 0$. Thus,
      $$ \int_0^{2\pi} (\sin(n_k x) - \sin(n_{k + 1} x))^2 dx = \int_0^{2\pi} dx = 2\pi. $$
      This implies that
      \begin{equation}\label{eq:2}
        \lim_{n \to \infty} \int_0^{2\pi} (\sin(n_k x) - \sin(n_{k + 1} x))^2 dx = 2\pi.
      \end{equation}
      \eqref{eq:1} contradicts \eqref{eq:2}.
    \end{proof}
  \end{enumerate}
\item $f_n(x) = \frac{x^2}{x^2 + (1 - nx)^2}, x \in [0, 1], n = 1, 2, 3, \ldots$.
  \begin{enumerate}
  \item $|f_n(x)| \leq \frac{x^2}{x^2} = 1 < 2$. Thus, $\{ f_n \}$ is uniformly bounded.
  \item We claim that no subsequences of $\{ f_n \}$ converges uniformly on $[0, 1]$.
    \begin{proof}
      Recall that $\{ f_n \}$ converges uniformly to a function $f$ on $E$ if for every $\epsilon > 0$, there exists an integer $N$ such that $n \geq N$ implies $|f_n(x) - f(x)| < \epsilon$ for all $x \in E$.

      The negation of this definition states that $\{ f_n \}$ does not converge uniformly to $f$ on $E$ if there exists $\epsilon_0 > 0$ such that for every integer $n$, there exists $\mathcal N_0 \geq N$ such that $|f_{N_0}(x_0)  - f(x_0)| \geq \epsilon_0$ for some $x_0 \in E$.

      Note that $\lim_{n \to \infty} f_n(x) = 0$ for $0 \leq x \leq 1$. Let $f(x) = 0$ for $x \in [0, 1]$. Note that $f_n\left(\frac{1}{n}\right) = \frac{x^2}{x^2} = 1$. Take $\epsilon = \frac{1}{2}$ and $x_0 = \frac{1}{\mathcal N_0}$ for large $\mathcal N_0$. Thus,
      $$ \left|f_{\mathcal N_0}\left(\frac{1}{\mathcal N_0}\right) - f\left(\frac{1}{\mathcal N_0}\right)\right| = |1 - 0| = 1 \geq \epsilon_0 = \frac{1}{2}. $$
      Hence, $\{ f_n \}$ does not converge uniformly on $[0, 1]$. This proves that for any $\{ n_k \} \subseteq \NN$, $\{ f_{n_k} \}$ does not converge uniformly on $[0, 1]$.
    \end{proof}
  \end{enumerate}
\end{enumerate}

\subsection{Equicontinuous Families of Functions}

\begin{defn}
  \normalfont A family $\mathcal F$ of complex functions defined on a set $E$, in a metric space $X$, is said to be {\bf equicontinuous} on $E$ if for every $\epsilon > 0$, there exists $\delta > 0$ such that $|f(x) - f(y)| < \epsilon$ whenever $x \in E$, $y \in E$ and $f \in \mathcal F$.
\end{defn}

\begin{remark}
  \normalfont Every member of an equicontinuous family is uniformly continuous.
\end{remark}

\subsection{Diagonal Principle}

\begin{thm}[Diagonal principle]
  \normalfont If $\{ f_n \}$ is a pointwise bounded sequence of complex functions on a countable set $E$, then $\{ f_n \}$ has a subsequence $\{ f_{n_k} \}$ such that $\{ f_{n_k}(x) \}$ converges for every $x \in E$.
\end{thm}

\begin{proof}[Proof sketch]
  \renewcommand{\qedsymbol}{}
  Let $\{ x_i \}_{i = 1, 2, 3, \ldots}$ be the points of $E$.

  {\bf Step 1.} Since $\{ f_n(x_1) \}$ is bounded, there exists a subsequence $\{ f_{1, k_i} \}$ such that $\{ f_{1, k_i}(x_1) \}$ converges as $k \to \infty$.

  {\bf Step 2.} Since $\{ f_{1, k_i}(x_2) \}$ is bounded, there exists a subsequence $\{ f_{2, k_{i_j}} \}$ such that $\{ f_{2, k_{i_j}}(x_2) \}$ converges as $k \to \infty$.

  Repeat this process.

  $$
  \begin{array}{lllll}
    \textbf{Step 1:} & \text{\color{blue} $f_{1, k_1}$} & f_{1, k_2} & f_{1, k_3} & \ldots \\
    \textbf{Step 2:} & f_{2, k_{i_1}} & \text{\color{blue} $f_{2, k_{i_2}}$} & f_{2, k_{i_3}} & \ldots \\
    \textbf{Step 3:} & f_{3, k_{i_{j_1}}} & f_{3, k_{i_{j_2}}} & \text{\color{blue} $f_{3, k_{i_{j_3}}}$} & \ldots \\
    \vdots & & & &
  \end{array}
  $$
\end{proof}

\end{document}
