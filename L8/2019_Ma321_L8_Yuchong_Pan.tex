
\documentclass[letterpaper, reqno,11pt]{article}
\usepackage[margin=1.0in]{geometry}
\usepackage{color,latexsym,amsmath,amssymb}
\usepackage{fancyhdr}
\usepackage{amsthm}
\usepackage{mathtools}
\usepackage{tikz}
\usepackage{float}
\usepackage{centernot}
\usepackage{subcaption}
\usepackage{extarrows}
\usetikzlibrary{hobby}
\usetikzlibrary{shapes.multipart}
\usepackage{pgfplots}
\pgfplotsset{compat=1.7}

\newcommand{\RR}{\mathbb{R}}
\newcommand{\CC}{\mathbb{C}}
\newcommand{\ZZ}{\mathbb{Z}}
\newcommand{\QQ}{\mathbb{Q}}
\newcommand{\NN}{\mathbb{N}}
\DeclareMathOperator{\card}{card}
\DeclareMathOperator{\Binomial}{Binomial}
\pagestyle{fancy}
\lhead{Math 321 Lecture 8}
\rhead{Yuchong Pan}
\begin{document}
\pagenumbering{arabic}
\title{Math 321 Lecture 8}
\author{Yuchong Pan}
\date{January 18, 2019}
\newtheorem{thm}{Theorem}
\newtheorem{defn}{Definition}
\newtheorem*{remark}{Remark}
\newtheorem{claim}{Claim}
\newtheorem{cor}{Corollary}
\newtheorem{lemma}{Lemma}
\maketitle
%

\section{Proof of Arzela-Ascoli Theorem (Cont'd)}

\begin{proof}[Proof (cont'd)]
  Let $(X, d)$ be a compact metric space.

  ~

  \noindent {\bf Step 2:} Assume $\mathcal F \subseteq C(X)$ is closed, uniformly bounded and equicontinuous. We need to show that $\mathcal F$ is compact.

  ~
  
  {\centering (A) \par}
  
  \noindent\fbox{\begin{minipage}{\textwidth}
      $X$ compact $\Rightarrow$ $X$ separable
      
      i.e., $X$ admits a countable {\bf dense set} {\color{red} $Y$}.
  \end{minipage}}

  \noindent $X$ totally bounded $\Rightarrow$ $X \subseteq \bigcup_{i = 1}^{K_n} B\left(x_i^{(n)}; \frac{1}{n}\right)$. $Y = \bigcup_{n = 1}^\infty \left\{ x_i^{(n)} ; 1 \leq i \leq K_n \right\}$ is countable and dense.

  ~
  
  {\centering (B) \par}

  \noindent\fbox{\begin{minipage}{\textwidth}
      Suppose $\{ f_n ; n \geq 1 \}$ is a collection of functions on any metric space $X$, with the property that $|f_n(x)| \leq M$ for all $n \geq 1$ and for all $x \in X$. Let $\text{\color{red} $Y$} \subseteq X$ be any countable subset. Then there exists a subsequence $n_k \nearrow \infty$ such that $\{ f_{n_k} (y) ; k \geq 1 \}$ converges to some limit, for every $y \in Y$.
  \end{minipage}}

  \noindent (HW 1, problem 1)

  ~

  \noindent {\bf Outline of proof:} Recall (Math 320) that a set $A$ in a metric space $\mathcal A$ is compact if and only if every sequence $\{ a_n ; n \geq 1 \} \subseteq A$ admits a convergent subsequence $\{ a_{n_k} ; k \geq 1\}, a_{n_k} \xrightarrow{k \to \infty} a \in A$. Accordingly, pick any sequence $\{ f_n ; n \geq 1 \} \subseteq \mathcal F$. Our job is to find a subsequence $\{ n_k ; k \geq 1 \} \subseteq \{ f_n ; n \geq 1 \}$ such that $f_{n_k} \xrightarrow{k \to \infty} f$ uniformly on $X$.

  ~

  \noindent {\bf Claim:} (A) and (B) $\Rightarrow$

  {\centering (C) \par}

  \noindent\fbox{\begin{minipage}{\textwidth}
      The subsequence $\{ f_{n_k} ; k \geq 1 \}$ is uniformly Cauchy on $X$; i.e., $\lVert f_{n_k} - f_{n_{k'}} \rVert_\infty \xrightarrow{k, k' \to \infty} 0$.
  \end{minipage}}

  ~

  {\centering $\big\Downarrow{\text{$C(X)$ is compact}}$ \par}

  ~

  \noindent\fbox{\begin{minipage}{\textwidth}
      There exists $f \in C(X)$ such that $f_{n_k} \xrightarrow{k \to \infty} \underbrace{f}_\text{a limit point of $\mathcal F$}$ uniformly on $X$.

      $f \in \mathcal F$ because $\mathcal F$ is closed, hence contains all its limit points.
  \end{minipage}}

  ~

  \noindent {\bf Proof of claim:} (A) and (B) $\Rightarrow$ (C).

  Start with any $\epsilon > 0$. We need to find $K_0 = K_0(\epsilon) \geq 1$ such that
  $$ \lVert f_{n_k} - f_{n_{k'}} \rVert_\infty = \sup_{x \in X} |f_{n_k}(x) - f_{n_{k'}}(x)| < \epsilon ~ \forall k, k' \geq K_0. $$
  {\bf By equicontinuity of $\mathcal F$}, there exists $\delta > 0$ such that
  \begin{equation} \label{eq:star} \tag{*}
    d(x, x') < \delta \Rightarrow |g(x) - g(x')| < \frac{\epsilon}{3} ~ \forall g \in \mathcal F.
  \end{equation}
  Recall $X$ is compact, hence totally bounded. So we can cover $X$ by finitely many balls of radius $\delta$; i.e., $y_1, y_2, \ldots, y_R \in Y$ such that
  $$ X \subseteq \bigcup_{i = 1}^R B(y_i ; \delta). $$
  Given any $x \in X$, there exsts $y_i \in Y, 1 \leq i \leq R$ such that $d(x, y_i) < \delta$. Thus,
  $$ |f_{n_k}(x) - f_{n_{k'}}(x)| \leq \underbrace{|f_{n_k}(x) - f_{n_k}(y_i)|}_\text{I} + \underbrace{|f_{n_k}(y_i) - f_{n_{k'}}(y_i)|}_\text{II} + \underbrace{|f_{n_{k'}}(y_i) - f_{n_{k'}}(x)|}_\text{III}. $$
  By \eqref{eq:star}, I and III are each bounded above by $\frac{\epsilon}{3}$, provided $d(x, y) < \delta$.

  We want to choose $y \in Y$ and use the pointwise convergence of $\{ f_{n_k}(y) ; k \geq 1 \}$. Since (B) implies that $\{ f_{n_k}(y) ; k \geq 1 \}$ is convergent for every $y \in Y$, we have $\{ f_{n_k}(y) ; k \geq 1 \}$ is Cauchy for all $y \in Y$; i.e., given any $\epsilon > 0$, there exists $K = K(y, \epsilon) \geq 1$ such that $|f_{n_k}(y) - f_{n_{k'}}(y)| < \frac{\epsilon}{3}$ for all $k, k' \geq K(y, \epsilon)$. Thus, $\text{II} < \frac{\epsilon}{3}$ provided $k, k' \geq K_0 \stackrel{\text{def}}{=} \max(K(y_1, \epsilon), \ldots, K(y_R, \epsilon))$.
\end{proof}

\end{document}
