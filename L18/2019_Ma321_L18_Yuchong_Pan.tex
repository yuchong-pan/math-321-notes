
\documentclass[letterpaper, reqno,11pt]{article}
\usepackage[margin=1.0in]{geometry}
\usepackage{color,latexsym,amsmath,amssymb}
\usepackage{fancyhdr}
\usepackage{amsthm}
\usepackage{mathtools}
\usepackage{tikz}
\usepackage{float}
\usepackage{centernot}
\usepackage{subcaption}
\usepackage{extarrows}
\usetikzlibrary{hobby}
\usetikzlibrary{shapes.multipart}
\usepackage{pgfplots}
\pgfplotsset{compat=1.7}
\usetikzlibrary{arrows.meta}

\newcommand{\RR}{\mathbb{R}}
\newcommand{\CC}{\mathbb{C}}
\newcommand{\ZZ}{\mathbb{Z}}
\newcommand{\QQ}{\mathbb{Q}}
\newcommand{\NN}{\mathbb{N}}
\def\upint{\mathchoice%
  {\mkern13mu\overline{\vphantom{\intop}\mkern7mu}\mkern-20mu}%
  {\mkern7mu\overline{\vphantom{\intop}\mkern7mu}\mkern-14mu}%
  {\mkern7mu\overline{\vphantom{\intop}\mkern7mu}\mkern-14mu}%
  {\mkern7mu\overline{\vphantom{\intop}\mkern7mu}\mkern-14mu}%
  \int}
\def\lowint{\mkern3mu\underline{\vphantom{\intop}\mkern7mu}\mkern-10mu\int}
\DeclareMathOperator{\card}{card}
\DeclareMathOperator{\Binomial}{Binomial}
\pagestyle{fancy}
\lhead{Math 321 Lecture 18}
\rhead{Yuchong Pan}
\begin{document}
\pagenumbering{arabic}
\title{Math 321 Lecture 18}
\author{Yuchong Pan}
\date{February 13, 2019}
\newtheorem{thm}{Theorem}
\newtheorem{defn}{Definition}
\newtheorem*{remark}{Remark}
\newtheorem{claim}{Claim}
\newtheorem{cor}{Corollary}
\newtheorem{lemma}{Lemma}
\newtheorem{prop}{Proposition}
\maketitle
%

\section{Functions of Bounded Variation}

\subsection{Jordan's Theorem}

\begin{defn}
  \normalfont $\alpha : [a, b] \to \RR$ is in $BV[a, b]$ if $V_a^b \alpha = \sup_\text{$P$ partition} V_a^b (\alpha, P) = \sup_P \sum_{i = 1}^n |\alpha(x_i) - \alpha(x_{i - 1})| < \infty$.
\end{defn}

\begin{thm}[Jordan's theorem]
  \normalfont $\alpha \in BV[a, b]$ if and only if $\alpha$ can be written as $\alpha = \beta - \gamma$, with $\beta, \gamma$ nondecreasing.
\end{thm}

\begin{proof}
  ``$\Leftarrow$'': Assume $\alpha = \beta - \gamma$ with $\beta, \gamma \xrightarrow{\text{nondecreasing}} \RR$. Let $P = \{ a = x_0 < x_1 < \ldots < x_n = b \}$. Then,
  \begin{align*}
    V_a^b (\alpha, P) &= \sum_{i = 1}^n |\alpha(x_i) - \alpha(x_{i - 1})| \\
    &= \sum_{i = 1}^n |[\beta(x_i) - \beta(x_{i - 1})] - [\gamma(x_i) - \gamma(x_{i - 1})]| \\
    &\underbrace{\leq \sum_{i = 1}^n |\beta(x_i) - \beta(x_{i - 1})| + \sum_{i = 1}^n |\gamma(x_i) - \gamma(x_{i - 1})|}_\text{triangular inequality} \\
    &= V_a^b (\beta, P) + V_a^b (\gamma, P) \leq V_a^b \beta + V_a^b \gamma \\
    &= \underbrace{[\beta(b) - \beta(a)] + [\gamma(b) - \gamma(a)]}_\text{independent of $P$} < \infty.
  \end{align*}
  Hence, $V_a^b \alpha = \sup_P V_a^b (\alpha, P) \leq [\beta(b) - \beta(a)] + [\gamma(b) - \gamma(a)] < \infty$.

  \medskip

  \noindent {\bf Exercise:} If $f, g \in BV[a, b]$, then $V_a^b (f \pm g) leq V_a^b f + V_a^b g$.

  \medskip

  ``$\Rightarrow$'': Assume that $\alpha \in BV[a, b]$; need to find $\beta, \gamma$ nondecreasing such that $\underbrace{\alpha = \beta - \gamma}_{\Rightarrow ~ \gamma = \beta - \alpha}$. Introduce the variation function $v(x) = V_a^x \alpha = \text{total variation of $\alpha$ on $[a, x]$}$.

  \medskip

  \noindent {\bf Exercise:} $v(x)$ is well-defined because $V_a^b \alpha \geq V_a^c \alpha$ for any interval $[c, d] \subseteq [a, b]$.

  \medskip

  \noindent {\bf Question:} Is $v$ nondecreasing? Yes.

  Let $x < y$. Need to verify if $\underbrace{v(x)}_{= V_a^x \alpha} \leq \underbrace{v_y(y)}_{= V_a^y \alpha}$.

  True by Exercise above since $[a, x] \subseteq [a, y]$.

  \medskip

  \noindent {\bf Question:} Is $v(\cdot) - \alpha(\cdot)$ nondecreasing? Yes.

  Let $x < y$; need to verify if
  \begin{align*}
    v(x) - \alpha(x) &\leq v(y) - \alpha(y), \\
    \alpha(y) - \alpha(x) &\leq \underbrace{v(y) - v(x)}_{V_a^y \alpha - V_a^x \alpha = V_x^y \alpha = \sup_\text{$P$ partition of $[x, y]$} V_x^y (\alpha, P)}.
  \end{align*}

  \begin{figure}[H]
    \centering
    \begin{tikzpicture}
      \draw (0, 0) -- (5, 0);
      \draw (0, 0.1) -- (0, -0.1);
      \draw (1.5, 0.1) -- (1.5, -0.1);
      \draw (3, 0.1) -- (3, -0.1);
      \node[below] at (0, 0) {$a$};
      \node[above] at (1.5, 0) {$x$};
      \node[above] at (3, 0) {$y$};
      \draw (0, 0.5) to[bend left=50] (1.5, 0.5);
      \draw (0, -0.5) to[bend right=50] (3, -0.5);
      \draw[dashed, red, thick] (1.5, 0) -- (3, 0);
    \end{tikzpicture}
  \end{figure}

  Note: $|\alpha(y) - \alpha(x)| = V_x^y \alpha(x, P_0)$, where $P_0 = \{ x, y \}$. Thus,
  $$ \alpha(y) - \alpha(x) \leq |\alpha(y) - \alpha(x)| = V_x^y (\alpha, P_0) \leq V_x^y \alpha = v(y) - v(x). $$

  \medskip

  \noindent {\bf Exercise:} Check that for any $c \in (a, b)$ and any $f \in BV[a, b]$,
  $$ V_a^c f + V_c^b f = V_a^b f. $$
\end{proof}

\begin{remark}
  \normalfont Jordan's theorem suggests that $BV[a, b]$ could be a good source of Riemann-Stieltjes integrators, with respect to which continuous functions on $[a, b]$ can be integrated.
  $$ \int fd\underbrace{\alpha}_{\in BV} = \int fd(\beta - \gamma) \xlongequal{\text{?}} \int fd\beta - \int fd\gamma. $$
\end{remark}

\subsection{Interchanging Integrands and Integrators}

\noindent {\bf Recall:}
$$ \int_a^b (\underbrace{udv}_{=u(x)v'(x)dx} + \underbrace{vdu}_{=v(x)u'(x)dx}) = u(b) v(b) - u(a) v(a). \qquad \longrightarrow \text{integration by parts} $$

\begin{thm}[Integration by parts for Riemann-Stieltjes integrals]
  \normalfont Let $f, \alpha \to \RR$ be arbitrary functions. Then,
  $$ \underbrace{f \in \mathcal R_\alpha[a, b]}_{\substack{\exists I \text{ s.t. } \forall \epsilon > 0, \exists \text{a partition $P_0$} \text{ s.t. } |S_\alpha(f, P, T) - I| < \epsilon \\ \text{ for $P \supseteq P_0$ and any selection of points $T$ subordinate to $P$}}} \Leftrightarrow \alpha \in \mathcal R_f[a, b]. $$

  In either case,
  $$ \int_a^b fd\alpha + \int_a^b \alpha df = \alpha(b) f(b) - \alpha(a) f(a). $$
\end{thm}

\begin{cor}
  \normalfont $C[a, b] \subseteq \mathcal R_\alpha[a, b]$ for all $\alpha \in BV[a, b]$.
\end{cor}

\begin{proof}
  Use Jordan's theorem to write $\alpha = \beta - \gamma$ with $\beta, \gamma$ nondecreasing.

  Note by a previous theorem, $C[a, b] \subseteq \mathcal R_\beta[a, b] \cap \mathcal R_\gamma[a, b]$. This implies that if $f \in C[a, b]$, then $\int_a^b fd\beta$ and $\int_a^b fd\gamma$ are well-defined.

  By IBP, $\int_a^b \beta df$ and $\int_a^b \gamma df$ are well-defined too; i.e., $\beta, \gamma \in \mathcal R_f[a, b]$. But it is easy to see that $\mathcal R_f[a, b]$ is a vector space, with
  $$ \int_a^b (\beta \pm \gamma) df = \int_a^b \beta df \pm \int_a^b \gamma df. $$
  This implies $\beta \pm \gamma \in \mathcal R_f[a,b]$; in particular, $\alpha = \beta - \gamma \in  \mathcal R_f[a, b]$.

  Use IBP again; get $f \in \mathcal R_\alpha[a, b]$.
\end{proof}

\end{document}
